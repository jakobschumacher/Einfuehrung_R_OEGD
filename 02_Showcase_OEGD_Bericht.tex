% Options for packages loaded elsewhere
\PassOptionsToPackage{unicode}{hyperref}
\PassOptionsToPackage{hyphens}{url}
%
\documentclass[
]{article}
\usepackage{amsmath,amssymb}
\usepackage{iftex}
\ifPDFTeX
  \usepackage[T1]{fontenc}
  \usepackage[utf8]{inputenc}
  \usepackage{textcomp} % provide euro and other symbols
\else % if luatex or xetex
  \usepackage{unicode-math} % this also loads fontspec
  \defaultfontfeatures{Scale=MatchLowercase}
  \defaultfontfeatures[\rmfamily]{Ligatures=TeX,Scale=1}
\fi
\usepackage{lmodern}
\ifPDFTeX\else
  % xetex/luatex font selection
\fi
% Use upquote if available, for straight quotes in verbatim environments
\IfFileExists{upquote.sty}{\usepackage{upquote}}{}
\IfFileExists{microtype.sty}{% use microtype if available
  \usepackage[]{microtype}
  \UseMicrotypeSet[protrusion]{basicmath} % disable protrusion for tt fonts
}{}
\makeatletter
\@ifundefined{KOMAClassName}{% if non-KOMA class
  \IfFileExists{parskip.sty}{%
    \usepackage{parskip}
  }{% else
    \setlength{\parindent}{0pt}
    \setlength{\parskip}{6pt plus 2pt minus 1pt}}
}{% if KOMA class
  \KOMAoptions{parskip=half}}
\makeatother
\usepackage{xcolor}
\usepackage[margin=1in]{geometry}
\usepackage{graphicx}
\makeatletter
\def\maxwidth{\ifdim\Gin@nat@width>\linewidth\linewidth\else\Gin@nat@width\fi}
\def\maxheight{\ifdim\Gin@nat@height>\textheight\textheight\else\Gin@nat@height\fi}
\makeatother
% Scale images if necessary, so that they will not overflow the page
% margins by default, and it is still possible to overwrite the defaults
% using explicit options in \includegraphics[width, height, ...]{}
\setkeys{Gin}{width=\maxwidth,height=\maxheight,keepaspectratio}
% Set default figure placement to htbp
\makeatletter
\def\fps@figure{htbp}
\makeatother
\setlength{\emergencystretch}{3em} % prevent overfull lines
\providecommand{\tightlist}{%
  \setlength{\itemsep}{0pt}\setlength{\parskip}{0pt}}
\setcounter{secnumdepth}{-\maxdimen} % remove section numbering
\ifLuaTeX
  \usepackage{selnolig}  % disable illegal ligatures
\fi
\IfFileExists{bookmark.sty}{\usepackage{bookmark}}{\usepackage{hyperref}}
\IfFileExists{xurl.sty}{\usepackage{xurl}}{} % add URL line breaks if available
\urlstyle{same}
\hypersetup{
  pdftitle={ÖGD Bericht für Hessen},
  hidelinks,
  pdfcreator={LaTeX via pandoc}}

\title{ÖGD Bericht für Hessen}
\author{}
\date{\vspace{-2.5em}2023-04-27}

\begin{document}
\maketitle

\begin{figure}
\centering
\includegraphics{logo2.png}
\caption{Logo}
\end{figure}

\hypertarget{einleitung}{%
\section{Einleitung}\label{einleitung}}

Dies ist ein Bericht für den öffentlichen Gesundheitsdienst. Er ist
erstellt mit R und RStudio. Inhaltlilch dreht sich der Bericht um
Covid-19-Hospitalisierungen

\hypertarget{methoden}{%
\section{Methoden}\label{methoden}}

\hypertarget{datenquelle}{%
\subsection{Datenquelle}\label{datenquelle}}

Die Daten wurden heruntergeladen von der Opendata-Webseite des RKI.

\begin{itemize}
\tightlist
\item
  Datenquelle:
  \url{https://raw.githubusercontent.com/robert-koch-institut/COVID-19-Hospitalisierungen_in_Deutschland/master/Aktuell_Deutschland_adjustierte-COVID-19-Hospitalisierungen.csv}
\end{itemize}

\hypertarget{vorgehensweise}{%
\subsection{Vorgehensweise}\label{vorgehensweise}}

Bei den Methoden wurde wie folgt vorgegangen:

\begin{enumerate}
\def\labelenumi{\arabic{enumi}.}
\tightlist
\item
  Daten einlesen
\item
  Daten transformieren
\item
  Berechnungen anstellen
\item
  Grafiken erstellen
\end{enumerate}

\hypertarget{scriptinformationen}{%
\subsection{Scriptinformationen}\label{scriptinformationen}}

\hypertarget{ergebnisse}{%
\section{Ergebnisse}\label{ergebnisse}}

Insgesamt gibt es 4726 Datenpunkte.

\includegraphics{02_Showcase_OEGD_Bericht_files/figure-latex/abbildung_1-1.pdf}

\end{document}
